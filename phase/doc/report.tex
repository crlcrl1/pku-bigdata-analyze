\documentclass{article}

\usepackage{neurips_2022}
\usepackage[utf8]{inputenc}
\usepackage[T1]{fontenc}
\usepackage{hyperref}
\usepackage{url}
\usepackage{booktabs}
\usepackage{amsfonts}
\usepackage{nicefrac}
\usepackage{microtype}
\usepackage{xcolor}
\usepackage{amsmath}
\usepackage{amssymb}
\usepackage[UTF8, fontset=adobe]{ctex}
\usepackage{algorithm}
\usepackage{algpseudocode}
\usepackage{resizegather}
\usepackage{graphicx}
\usepackage{subfigure}
\usepackage{flafter}

\hypersetup{
    colorlinks=true,
    linkbordercolor=white
}
\setlength{\parindent}{2em}

\title{相位恢复上机报告}

\author{陈润璘 \texttt{2200010848}}

\begin{document}

\maketitle

\section{问题描述}

相位恢复问题是指从信号的幅值(强度)测量中恢复其相位信息。在数学上,给定向量 $x \in \mathbb{C}^n$,我们只能观测到
$y_j = |\left<a_j, x\right>|^2$,其中 $a_j \in \mathbb{C}^{n}$
是一个已知矩阵,$y_j$ 是测量值。我们的目标是恢复原始信号 $x$(或其相位等价形式)。

本次上机作业中,我们考虑两种情形:
\begin{enumerate}
    \item 高斯随机测量:测量矩阵 $A = (a_1, \dots , a_m)$ 的元素是从复高斯分布中随机抽取的。
    \item 编码衍射图样(Coded Diffraction Pattern,CDP):使用特定的掩模结构进行测量。
\end{enumerate}

\section{数值方法}

为了解决相位恢复问题,本实验采用基于绝对最小二乘的次梯度下降方法。我们考虑优化目标函数:

\begin{equation}
    f(z) = \frac{1}{m}\sum_{j=1}^{m}||\left<a_j, x\right>|^2 - y_j|
\end{equation}

其中 $z$ 是我们要优化的估计向量,$a_j$ 是测量向量,$y_j$ 是观测到的强度值。

\subsection{初始化策略}

由于优化问题是非凸的,因此良好的初始值对于非凸优化问题至关重要。我们可以选取初始值为
\begin{equation}
    Y = \frac{1}{m}\sum_{j=1}^{m} y_j \cdot a_j a_j^*
\end{equation}
的最大特征值对应的特征向量,由此得到一个相位接近真实值的初始解,其中最大特征值可以通过幂法求得。

\subsection{次梯度下降算法}

我们采用次梯度下降法优化目标函数,算法流程如下:

\begin{algorithm}
    \caption{绝对最小二乘相位恢复算法}
    \begin{algorithmic}[1]
        \State 使用上述方法初始化 $z$
        \State 初始化最佳值 best\_value $\gets \infty$,最佳解 best\_z $\gets$ z
        \For{$k = 0, 1, \ldots, \text{num\_iter}-1$}
        \State 计算当前目标函数值 current\_obj\_val
        \If{current\_obj\_val < best\_value}
        \State best\_value $\gets$ current\_obj\_val
        \State best\_z $\gets$ z
        \EndIf
        \State 计算 $s = \mathcal{A}(z)$
        \State 计算残差 $g = |s|^2 - y$
        \State 计算梯度 $\nabla f(z) = \frac{1}{m} \cdot
        \mathcal{A}^*(\text{sign}(g) \cdot 2s)$
        \State 根据迭代轮数 $k$ 确定步长 $\eta_k$
        \State 更新 $z \gets z - \eta_k \cdot \nabla f(z)$
        \EndFor
        \State 返回 best\_z
    \end{algorithmic}
\end{algorithm}

本实验中,我们对两种不同的测量矩阵(高斯随机矩阵和CDP)采用了不同的步长调度策略:

\begin{itemize}
    \item 高斯情形:前200次迭代 $\eta_k = \frac{1}{k+1}$,之后 $\eta_k =
        \frac{1}{(k+1)^{1.3}}$
    \item CDP情形:前2000次迭代 $\eta_k = \frac{1}{(k+1)^{0.3}}$,之后 $\eta_k
        = \frac{1}{k+1}$
\end{itemize}

\subsection{评估指标}

由于相位恢复问题只能恢复信号的相位等价类,我们使用相对误差来评估恢复质量:

\begin{equation}
    \text{rel\_error}(x, z) = \frac{\|x - e^{-i\angle\langle x,
    z\rangle} \cdot z\|_2}{\|x\|_2}
\end{equation}

其中 $\angle\langle x, z\rangle$ 是 $x$ 和 $z$ 内积的相角,$e^{-i\angle\langle
x, z\rangle}$ 表示对 $z$ 进行相位校正。

\section{数值结果}

\subsection{高斯随机测量}

在高斯随机测量情况下,我们设置信号维度 $n = 128$,测量数量 $m \approx 4.5n = 576$。
测量矩阵 $A \in\mathbb{C}^{m \times n}$ 的元素从复高斯分布 $\mathcal{CN}(0,
\frac{1}{2})$ 中随机生成。

实验结果显示,绝对最小二乘方法在高斯随机测量情况下表现良好。我们使用5000次迭代,最终相对误差的大小为 $1.4585\times 10^{-4}$。
目标函数值随迭代次数的变化如图 \ref{fig:gaussian} 所示。
从图中可以看出,目标函数值随着迭代次数的增加而迅速下降,表明算法收敛效果良好。同时,目标函数值在迭代过程中呈现了次梯度法常见的震荡下降现象。

\begin{figure}[h]
    \centering
    \includegraphics[width=0.55\textwidth]{fig/gaussian.pdf}
    \caption{高斯随机测量下目标函数值随迭代次数的变化}
    \label{fig:gaussian}
\end{figure}

\subsection{编码衍射图样(CDP)}

在CDP情况下,我们同样设置信号维度 $n = 128$,但使用了特殊的测量结构。我们生成 $l = 6$ 个掩模,每个掩模中的元素从集合
$\{1, -1, i, -i\}$ 中随机选取,并根据一定概率($20\%$)赋予不同的权重($\sqrt{3}$ 或 $1/\sqrt{2}$)。

在 5000 次迭代后,最终相对误差的大小为 $1.1782\times 10^{-4}$。
目标函数值随迭代次数的变化如图 \ref{fig:cdp} 所示。
从图中可以看出,目标函数值在迭代过程中同样呈现了震荡下降的趋势,但收敛速度更快,且最终得到的相对误差较小。与高斯随机测量相比,CDP
方法通过设计特定的掩模结构,使用了更少的测量次数和计算代价并获得了更高的恢复精度。
\begin{figure}[h]
    \centering
    \includegraphics[width=0.55\textwidth]{fig/cdp.pdf}
    \caption{CDP下目标函数值随迭代次数的变化}
    \label{fig:cdp}
\end{figure}

\subsection{真实图像恢复}

除了随机生成的数据,我们还对一张真实图像进行了相位恢复实验。图像被转换为灰度图,并归一化至[0,1]范围内。
在这个实验中,我们使用了 30 个掩模,每个掩模的产生方式与CDP相同。

实验结果显示,即使在较少的迭代次数(1000次)下,算法也能有效地恢复图像。
最终的相对误差约为 $1.4878\times 10^{-4}$,而目标函数值约为 $2.7345$。
图 \ref{fig:image_opt} 显示了优化过程中的目标函数值变化,
图 \ref{fig:reconstructed} 显示了原始图像与重建图像的对比。

\begin{figure}[h]
    \centering
    \subfigure[优化过程中的目标函数值]{
        \includegraphics[width=0.45\textwidth]{fig/real_image.pdf}
        \label{fig:image_opt}
    }
    \subfigure[原始图像与重建图像的对比]{
        \includegraphics[width=0.45\textwidth]{fig/real_image_reconstructed.pdf}
        \label{fig:reconstructed}
    }
    \caption{真实图像的相位恢复结果}
    \label{fig:real_image}
\end{figure}

从重建效果来看,算法能够很好地保留图像的主要特征和细节,这表明我们的相位恢复方法不仅适用于理论数据,也能处理实际的图像恢复问题。

\end{document}
