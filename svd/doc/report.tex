\documentclass{article}

\usepackage{neurips_2022}
\usepackage[utf8]{inputenc}
\usepackage[T1]{fontenc}
\usepackage{hyperref}
\usepackage{url}
\usepackage{booktabs}
\usepackage{amsfonts}
\usepackage{nicefrac}
\usepackage{microtype}
\usepackage{xcolor}
\usepackage{amsmath}
\usepackage{amssymb}
\usepackage[UTF8, fontset=adobe]{ctex}
\usepackage{algorithm}
\usepackage{algpseudocode}
\usepackage{resizegather}
\usepackage{graphicx}
\usepackage{subfigure}

\hypersetup{
    colorlinks=true,
    linkbordercolor=white
}
\setlength{\parindent}{2em}

\title{随机奇异值分解算法上机报告}

\author{陈润璘 \texttt{2200010848}}

\begin{document}

\maketitle

\section{问题描述}

奇异值分解是矩阵分析和数据降维中的重要工具。对于大规模矩阵,直接计算SVD的计算复杂度较高,
难以满足实际应用中对效率的要求。因此,设计高效地计算一个矩阵前 $k$ 个奇异值和奇异向量的近似算法显得尤为重要。

\section{数值方法}

本次上机作业实现了一种基于概率采样的 SVD 近似算法。其步骤如下:

\begin{algorithm}
    \caption{随机奇异值分解算法}
    \begin{algorithmic}[1]
        \State 输入矩阵 $A \in \mathbb{R}^{m \times n}$,目标秩 $k$,采样列数
        $c$,以及每一列的采样概率 $p$
        \State 以 $p_i$ 的概率采样第 $i$ 列,总共采样 $c$ 列,得到 $A^{i_1},
        A^{i_2},\dots,A^{i_c}$
        \State 以 $\frac{A^{i_t}}{\sqrt{cp_{i_t}}}$ 组成 $m\times c$ 的矩阵 $C$。
        \State 计算 $C^T C$ 的特征值分解,得到 $C^T C =
        \sum_{t=1}^{c}\sigma_t^2(C)y_t y_t^T$。
        \State 取前 $k$ 个特征值为 $A$ 的近似奇异值,其对应的左奇异向量为 $\frac{Cy_t}{\sigma_t(C)}$。
    \end{algorithmic}
    \label{alg:svd}
\end{algorithm}

上述算法仅需计算一个 $c\times c$ 的对称矩阵的特征值分解,当 $c$ 远小于 $m$ 和 $n$ 时,比直接计算一个
$m\times m$ 或 $n\times n$ 的矩阵的特征值分解计算量要小得多。

对于每一列的采样概率 $p$ 的选取,可以取 $p_i =
\frac{\|A^i\|_2^2}{\sum_{j=1}^{n}\|A^j\|_2^2}$。在这种选取下,$C C^T$ 能最好地近似
$A A^T$,从而 $C^T C$ 与 $A^T A$ 的非零特征值最接近。

\section{数值结果}

在本次上机作业中,我们分别使用一个秩为 20 的随机生成的 $2048\times 512$ 的矩阵和一张 $4096\times
3072$ 的图片作为输入矩阵,分别取 $c = 50, 100, 150, 200$,绘制了在 $c$ 的不同取值下计算得出的前 $k$
个奇异值的相对误差和下标 $k$ 的分布图。其中,$2048\times 512$ 的矩阵由两个元素为标准正态分布的
$2048\times 20$ 和 $20\times 512$ 的矩阵相乘得到,$4096\times 3072$ 的图片由原图转换为灰度图后得到。

\begin{figure}[ht]
    \centering
    \subfigure[随机生成的 $2048\times 512$ 的矩阵的结果]{
        \includegraphics[width=0.45\textwidth]{fig/relative_error_matrix.pdf}
    }
    \subfigure[$4096\times 3072$ 的图片的结果]{
        \includegraphics[width=0.45\textwidth]{fig/relative_error_img.pdf}
    }
    \caption{随机奇异值分解算法的数值结果}
    \label{fig:svd_result}
\end{figure}

从图 \ref{fig:svd_result} 中可以看出,随着 $c$ 的增大,计算得到的前 $k$
个奇异值的相对误差逐渐减小,但是两个算例中误差随 $k$ 的变化却呈现了不同的规律。对于随机生成的 $2048\times 512$
的矩阵,第 $k$ 个奇异值的相对误差随着 $k$ 的增大先减小,再增大,在 $k=9$ 附近达到最小值。而对于 $4096\times
3072$ 的图片,第 $k$ 个奇异值的相对误差随着 $k$ 的增大呈现增大的趋势。这可能是因为两个矩阵的结构不同,导致了误差的变化情况不同。

\end{document}
