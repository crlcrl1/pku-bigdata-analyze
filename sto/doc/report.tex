\documentclass{article}

\usepackage{neurips_2022}
\usepackage[utf8]{inputenc}
\usepackage[T1]{fontenc}
\usepackage{hyperref}
\usepackage{url}
\usepackage{booktabs}
\usepackage{amsfonts}
\usepackage{nicefrac}
\usepackage{microtype}
\usepackage{xcolor}
\usepackage{amsmath}
\usepackage{amssymb}
\usepackage[UTF8, fontset=adobe]{ctex}
\usepackage{algorithm}
\usepackage{algpseudocode}
\usepackage{resizegather}
\usepackage{graphicx}
\usepackage{subfigure}

\hypersetup{
    colorlinks=true,
    linkbordercolor=white
}
\setlength{\parindent}{2em}

\title{随机优化算法上机报告}

\author{陈润璘 \texttt{2200010848}}

\begin{document}

\maketitle

\section{背景与目的}
随机优化算法(如Adagrad与Adam)在大规模机器学习任务中具有重要应用。本次上机作业实现了 Adagrad 和 Adam 优化器,对比并分析了其在模型训练中的表现。

\section{算法原理}
\subsection{Adagrad}
Adagrad是一种自适应学习率优化算法,其对每个参数维护一个累计平方梯度的和,根据历史梯度自动调整学习率。其参数更新公式为:
\begin{equation}
    \theta_{t+1} = \theta_t - \frac{\eta}{\sqrt{G_t + \epsilon}} \odot g_t
\end{equation}
其中$G_t$为累计平方梯度,$\epsilon$ 为平滑项。

\subsection{Adam}
Adam 结合了动量法和 RMSProp 的思想,分别对一阶矩和二阶矩进行估计。其参数更新步骤如下:
\begin{align}
    m_t &= \beta_1 m_{t-1} + (1-\beta_1)g_t \\
    v_t &= \beta_2 v_{t-1} + (1-\beta_2)g_t^2 \\
    \hat{m}_t &= \frac{m_t}{1-\beta_1^t} \\
    \hat{v}_t &= \frac{v_t}{1-\beta_2^t} \\
    \theta_{t+1} &= \theta_t - \frac{\eta}{\sqrt{\hat{v}_t}+\epsilon} \hat{m}_t
\end{align}

\section{实验设计}
\begin{itemize}
    \item \textbf{数据集:} 使用 LIBSVM 中的 \texttt{covtype} 和 \texttt{gisette} 数据集,它们是两个分别有 54 和 5000 特征的二分类数据集。
    \item \textbf{模型结构:} 线性分类器,损失函数为 $1-\tanh(y\cdot f(x))$,同时添加 $L_2$ 正则项。
    \item \textbf{优化器:} 实现Adagrad和Adam两种优化器,分别训练模型。
    \item \textbf{评估指标:} 训练误差、测试误差与测试集准确率。
\end{itemize}

\section{实验结果与分析}

本次上机作用基于 PyTorch 框架实现了 Adagrad 和 Adam 优化器,具体的超参数见 \texttt{code/configs} 中的配置文件。在训练过程中,两个优化器使用了相同的 batch size 和正则化参数,并且训练相同数量的 epoch。两个优化器在两个数据集上的训练结果如下:

\begin{figure}[h]
    \centering
    \subfigure[Adagrad on covtype]{
        \includegraphics[width=0.45\textwidth]{fig/covtype-adagrad.pdf}
    }
    \subfigure[Adam on covtype]{
        \includegraphics[width=0.45\textwidth]{fig/covtype-adam.pdf}
    }
    \caption{Adagrad 和 Adam 在 \texttt{covtype} 数据集上的训练和测试误差}
    \label{fig:covtype_results}
\end{figure}

\begin{figure}[h]
    \centering
    \subfigure[Adagrad on gisette]{
        \includegraphics[width=0.45\textwidth]{fig/gisette-adagrad.pdf}
    }
    \subfigure[Adam on gisette]{
        \includegraphics[width=0.45\textwidth]{fig/gisette-adam.pdf}
    }
    \caption{Adagrad 和 Adam 在 \texttt{gisette} 数据集上的训练和测试误差}
    \label{fig:gisette_results}
\end{figure}

在数据集 \texttt{covtype} 上,Adagrad 和 Adam 在测试集上的准确率分别为 $76.28\%$ 和 $76.47\%$,在数据集 \texttt{gisette} 上,Adagrad 和 Adam 的准确率分别为 $95.20\%$ 和 $97.20\%$。从模型在测试集上的准确率来看,Adam 优化器略优于 Adagrad。但是,从图 \ref{fig:gisette_results} 中可以看出,Adam 在训练时的稳定性明显高于 Adagrad,误差的波动幅度更小。这是由于 \texttt{gisette} 的规模相对较小,Adam 记录了更多的历史信息,因此受梯度噪声的影响更小。

\end{document}
